% \iffalse
% Copyright 2012-2022 Jiro Senju
%
% This package is free software; you can redistribute it and/or modify
% it under the terms of the GNU General Public License as published by
% the Free Software Foundation; either version 2 of the License, or
% any later version.
%
% This package is distributed in the hope that it will be useful,
% but WITHOUT ANY WARRANTY; without even the implied warranty of
% MERCHANTABILITY or FITNESS FOR A PARTICULAR PURPOSE.  See the
% GNU General Public License for more details.
%
% You should have received a copy of the GNU General Public License
% along with this package.  If not, see <http://www.gnu.org/licenses/>.
%
%<package>\NeedsTeXFormat{LaTeX2e}
%<package>\ProvidesPackage{dvisirule}%
%<package>[2022/06/27 v1.1 dvisirule]
%
%<*driver>
\documentclass[draft]{ltxdoc}
%
% erewhon looser option has a bug
\usepackage[scaled=.95,loosest]{erewhon}
\usepackage[erewhon,vvarbb,bigdelims]{newtxmath}
\usepackage[scale=.95]{Chivo}

\usepackage[dvipdfmx,final]{graphicx}
\usepackage[final]{listings}
\lstset{basicstyle=\small\ttfamily,
  language=TeX,
  columns=[c]flexible,
  tabsize=4,
  %frame=tb,
  keepspaces=true,
  %lineskip=-.1ex,
  belowskip=\medskipamount}
\usepackage{multicol}
\usepackage{needspace}
%
% hyperref should come last
\usepackage[dvipdfmx,final,pdfusetitle]{hyperref}
% CJKbookmarks=true
% bookmarksdepth=section
% pdffitwindow=true
% pdfpagetransition=Dissolve
% pdfstartview=FitB
\hypersetup{
  hyperfootnotes=false,
  colorlinks=true,
  linkcolor=blue,
  % anchorcolor=black,
  % citecolor=black,
  % urlcolor=black,
  bookmarks=true,
  bookmarksnumbered=true,
  setpagesize=false,
  %pdftitle={},
  %pdfauthor={},
  pdfpagelayout=SinglePage,
  pdfpagemode=UseOutlines,
  pdfstartview=FitH
}
%
\usepackage{\jobname}
%
\makeatletter
% \def\meta@font@select{\slshape}
% \def\fps@table{htbp}
\def\fps@figure{htbp}
%
\newcommand{\NoDescription}{\vspace*{-.9\baselineskip}}
%
% \EnableCrossrefs
\DisableCrossrefs
\CodelineIndex
% \RecordChanges
%
\begin{document}
\MakeShortVerb{\|}
\DocInput{\jobname.dtx}
\end{document}
%</driver>
% \fi
%
%\GetFileInfo{\jobname.sty}
%
% \title{{\spaceskip=1ex\textsf{\jobname} package \fileversion\\
%     (\texttt{sitabular} environment)}}
% \author{\texttt{jiro1010senju at gmail dot com}}
% \date{\filedate}
% \maketitle
%
% Superimpose the hidden\slash covered |\hline| and |\vline| in a LaTeX
% tabular\slash colortbl environment.
%
% %%%%%%%%%%%%%%%%%%%%%%%%%%%%%%%%%%%%%%%
% \section{Motivation}
%
% Over a decade ago, I need to write down some latex tables in color.
% Not so many colors used, just two background colors.
% One is white as a usual background, and the other is gray which is
% to be easier to distinguish the preceding and succeeding rows from each
% other.
% In other words, the tables have two different row colors one after the
% other.
%
% Fortunately we have |\rowcolors| from |xcolor.sty| and it is really
% helpful, except one thing.
% The problem is the lines (|\hline|, |\cline| and others) are not shown
% in the PDF viewer.
% Technically speaking, if we zoom in\slash out, then the lines appear,
% but not always.
% If I change the thickness of the lines, I can always see them.
% But it is too ugly.
% It seems highly depending on the monitor resolution and the PDF viewer's
% internal calculation of the coordinates.
% Even if I set the width of the lines to 4pt, it can be only one pixel on
% the screen.
% It is totally up to the environment of the viewer.
%
% To address this issue, I guessed the cause is the order of drawing
% lines and painting the row background color.
% E.g. drawing lines first, and then painting the background color next.
% In that case, the background color may hide the lines due to the
% internal rounding-up\slash down the calculated coordinates.
%
% Some TeXnicians may be able to modify the tabular\slash colortbl
% environment, and postpone drawing the lines after completing the
% background color.
% But I don't have such TeXnique.
% I had considered a few ways to get what I want,
% and my solution is to draw the lines again at the end of the page by a
% new dviware.
%
% %%%%%%%%%%%%%%%%%%%%%%%%%%%%%%%%%%%%%%%
% \section{Basic approach}
%
% \begin{enumerate}
% \item
% In |.tex| (actually |.sty|), embed the markers to represent the
% begin\slash end of a line.
%
% \item
% A new dviware searches the embedded markers, if found, remembers the
% coordinates in the DVI page of them, and the offset in the DVI file.
%
% \item
% When the dviware reaches the end of the page, it copies the DVI
% instructions from the marker to the end of the page.
% I mean appending the line instructions to the page.
% And replace the the markers by NOP at the original position, to keep
% silent other dviwares.
% \end{enumerate}
%
% Now the lines are re-drawn over the row background color, and we can
% always see them clearly.
%
% %%%%%%%%%%%%%%%%%%%%%%%%%%%%%%%%%%%%%%%
% \section{Compile \& Install}
%
% \begin{lstlisting}[gobble=2,xleftmargin=1em]
% $ cd $this_dir/src
% $ configure -q
% $ cd ..
% $ make -s Dir=/tmp
% \end{lstlisting}
%
% Then you will get these files.
%
% \begin{lstlisting}[gobble=2,xleftmargin=1em]
% /tmp/dvisirule
% /tmp/dvisirule-bin
% /tmp/dvisirule-expg.mk
% /tmp/dvisirule-marker.awk
% /tmp/dvisirule-pgnum.awk
% /tmp/dvisirule.sty
% \end{lstlisting}
%
% Now you can install them by
%
% \begin{lstlisting}[gobble=2,xleftmargin=1em]
% $ make InstallBase=/tmp/texmf-dist
% \end{lstlisting}
%
% Here |${InstallBase}| is a make-variable which is referred by other make-variables.
%
% \begin{lstlisting}[gobble=2,xleftmargin=1em]
% InstallBin ?= ${InstallBase}/bin
% InstallLib ?= ${InstallBase}/lib
% InstallSty ?= ${InstallBase}/lib/texinputs
% \end{lstlisting}
%
% and the files will be installed to these dirs.
%
% \begin{lstlisting}[gobble=2,xleftmargin=1em]
% ${InstallBin}/dvisirule
%
% ${InstallLib}/dvisirule-bin
% ${InstallLib}/dvisirule-expg.mk
% ${InstallLib}/dvisirule-marker.awk
% ${InstallLib}/dvisirule-pgnum.awk
%
% ${InstallSty}/dvisirule.sty
% \end{lstlisting}
%
% Please note that these installed dirs have to be recognized by
% |Kpathsea| since the main |sh| script |dvisirule| searches these
% sub-files by |kpsewhich(1)|.
%
% %%%%%%%%%%%%%%%%%%%%%%%%%%%%%%%%%%%%%%%
% \section{Usage}
%
% |dvisirule.sty| is provided, and |sitabular| environment is a wrapper
% of |tabular| (including |colortbl|. You should include |xcolor.sty|
% or something BEFORE this pkg).
% The tabular syntax are kept, so you can use it by simply replacing |tabular| by
% |sitabular|.
% But it is not a main part of this package.
% |sitabular| does only embedding the markers and stores the
% page number to an external file.
% The main part of this packages is |dvisirule| (shell script) and the
% internal command |dvisirule-bin| (C\: program).
% So the usage here is not only including |dvisirule.sty| and
% |\begin{sitabular}|, but also run |dvisirule| after LaTeX compilation.
%
% \begin{lstlisting}[gobble=2,xleftmargin=1em]
% (a.tex)
% \usepackage{dvisirule}
%
% \begin{sitabular} ... \end{sitabular}
%
% latex a.tex
% cp -p a.dvi a.dvi.save
% dvisirule a.dvi a-si.dvi
% mv a-si.dvi a.dvi
% \end{lstlisting}
%
% You don't want run |dvisirule| command unconditionally?\,
% Good.\,
% |dvisirule.sty| creates and writes a file the page numbers which
% contain the line to
% be superimposed, so when the size of that file is zero, you can
% skip |dvisirule| command.
%
% \begin{lstlisting}[gobble=2,xleftmargin=1em]
% latex a.tex
% if [ -s a.sirule ]
% then
%       cp -p a.dvi a.dvi.save
%       dvisirule a.dvi a-si.dvi
%       mv a-si.dvi a.dvi
% fi
% \end{lstlisting}
%
% %%%%%%%%%%%%%%%%%%%%%%%%%%%%%%%%%%%%%%%
% \clearpage
% \section{Demo}
%
% \lstinputlisting{/tmp/demoprint}
%
% \includegraphics{/tmp/demo.pdf}
% \bigskip
%
% Tested on TeX Live 2019.
% \clearpage
%
% %%%%%%%%%%%%%%%%%%%%%%%%%%%%%%%%%%%%%%%
% \section{Limitation}
%
% |\hline|\, supports |\hline\hline| sequence, which means if followed by
% another |\hline|, it inserts |\vskip\doublerulesep|.
% This feature doesn't work in |sitabular|. It results a single thick
% |\hline|.  Use |\hhline| (from |hhline.sty|) instead.
%
% %%%%%%%%%%%%%%%%%%%%%%%%%%%%%%%%%%%%%%%
% \section{\texttt{dvisirule} script and sub-programs}
%
% The main script runs |dvitype| and other
% commands internally. You may want to replace some commands by other
% language specific variants. For |dvitype| command, you can set a
% environment variable, |$DVITYPE|. If it is set, the main script uses
% |$DVITYPE| instead of |dvitype|. Here is an example.
%
% \begin{lstlisting}[gobble=2,xleftmargin=1em]
% DVITYPE=pdvitype
% export DVITYPE
% dvisirule a.dvi
% \end{lstlisting}
%
% There are two other commands in the same group, |dviselect|
% (|$DVISELECT|) and |dviconcat| (|$DVICONCAT|).
%
% Also the main script refers to |$TMPDIR| environment variable. It
% specifies a directory where the extracted DVI page(s) are stored temporarily.
% If it is not set, |/tmp| is used.
%
% \subsection{\texttt{dvisirule} --- main shell script}
%
% Syntax\,:
% \lstinputlisting[xleftmargin=2em]{/tmp/help.txt}
%
% \begin{enumerate}
% \item
%   Accepts |.dvi|, and searches |.sirule|.
%   If |.sirule| doesn't exist, exits with an error.
%   If it exists but the size is zero, then exits with a success.
% \item
%   By an AWK script, |dvisirule-pgnum.awk|, splits |.dvi| per page,
%   dividing those who contains |sitabular| and who doesn't.
%   Sequential pages who doesn't contain |sitabular| are grouped into a
%   single file.
% \item
%   (for each extracted page which contains |sitabular|)
%   \par
%   By an AWK script, |dvisirule-marker.awk| finds and extracts the
%   markers from the page and the DVI line instructions from it.
% \item
%   (for each extracted page which contains |sitabular|)
%   \par
%   By a binary program |dvisirule-bin|, copies the line instructions
%   surrounded by the markers to the end of the page, and replaces the
%   markers and the line instructions at the original position (in the
%   DVI page) by NOP instruction, to stop other dviwares complain about
%   the marker.
% \item
%   Concatenate all pages, and make the final |.dvi|.
% \end{enumerate}
%
% \subsection{\texttt{dvisirule-bin} --- sub binary}
%
% This command has two inputs, one is a DVI file and the other is the
% output from |dvisirule-marker.awk|.
% The AWK script parses an output of |dvitype| for a single DVI page, and
% extract and prints SETRULE\slash PUTRULE instructions surrounded by the
% markers along with the color, the coordinates, and the offset in the DVI
% file.
%
% |dvisirule-bin| command parses the output of the AWK script, and copies
% those SETRULE\slash PUTRULE instructions to the end of the page. The
% markers and the instructions at the original offset are replaced by NOP
% instruction, so that other dviwares don't complain about that such as
% ``Unknown special'' or something. At last, |dvisirule-bin| creates and
% writes a new DVI file.
%
% You may be wondering is the AWK script really necessary? Why doesn't
% |dvisirule-|\allowbreak|bin| parse the DVI file by itself\,?
% Good point. It's just because I'm lazy. I can understand if a single
% binary operates all these processing, then the required cost (including
% CPU time) will be smaller. The performance will not be bad. But it is a
% run-rime performance. The lazy of me wants the develop-time performance
% too.
%
% Roughly speaking, splitting and parsing for each DVI page, and copying
% the instructions should be done concurrently. If all were implemented by
% C including multi-threading, the technical hurdle would be rather high.
% So I chose scripting for splitting and parsing. It can be run
% concurrently by |"make -j NCPU"|. Since DVI is a binary format, I had to
% implement it by C. It is OK, and the single-threaded C program can be
% run in parallel by |"make -j NCPU"| too.
% This is my laziness.
%
% \StopEventually
%
% %%%%%%%%%%%%%%%%%%%%%%%%%%%%%%%%%%%%%%%
% \section{Implementation}
%
% \subsection{\texttt{.sirule} file}
%
% \DescribeMacro{\si@RulePageW}
% \DescribeMacro{\si@RulePage}
% Create a file and store the absolute page numbers which contain
% |sitabular|.
% \smallskip
%
%    \begin{macrocode}
\RequirePackage{zref-abspage}
%
\newwrite\si@RulePageW
\immediate\openout\si@RulePageW=\jobname.sirule%\relax
\newcommand{\si@RulePage}{%\immediate
  \write\si@RulePageW{\theabspage}%
}
%    \end{macrocode}
%
% \subsection{marker}
%
% \DescribeMacro{\si@bol}
% \DescribeMacro{\si@eol}
% Begin/End markers of a line.
% \smallskip
%
%    \begin{macrocode}
\newcounter{si@rule}
\newcommand{\si@bol}{%
  \stepcounter{si@rule}%
  \special{sirule BOL\space\thesi@rule}%
  \si@RulePage%
}
\newcommand{\si@eol}{%
  \si@RulePage%
  \special{sirule EOL\space\thesi@rule}%
  \addtocounter{si@rule}{-1}%
}
%    \end{macrocode}
%
% \subsection{\texttt{\textbackslash hline} and \texttt{\textbackslash vline}}
%
% \DescribeMacro{\si@hline}
% \DescribeMacro{\si@vline}
% Originally |\hline| issues |\futurelet| internally to support
% |\hline\hline| sequence. But |\si@hline| issues |\si@eol| just after
% the original |\hline|. So |\hline\hline| would not work
% expectedly. Use |\hhline| (|hhline.sty|) instead.
% \smallskip
%
%    \begin{macrocode}
\let\si@Ohline=\hline
\newcommand{\si@hline}{%
  \noalign{\si@bol}%
  \si@Ohline%
  \noalign{\si@eol}%
}
\let\si@Ovline=\vline
\newcommand{\si@vline}{%
  \si@bol%
  \si@Ovline%
  \si@eol%
}
%    \end{macrocode}
%
% \subsection{\texttt{\textbackslash hhline} if loaded}
%
% \DescribeMacro{\si@hline}
% \DescribeMacro{\si@vline}
% \NoDescription
% \smallskip
%
%    \begin{macrocode}
\newif\ifsi@hhline
\@ifundefined{hhline}{}{%
  \global\si@hhlinetrue%
  \global\let\si@Ohhline=\hhline%
  \newcommand{\si@hhline}[1]{%
    \noalign{\si@bol}%
    \si@Ohhline{#1}%
    \noalign{\si@eol}%
  }%
}
%    \end{macrocode}
%
% \subsection{\texttt{sitabular} environment}
%
% \DescribeEnv{sitabular}
% \smallskip
%
%    \begin{macrocode}
\@ifundefined{newcolumntype}{%
  \global\let\newcolumntype=\@gobbletwo%
}{}
\let\si@Otabular=\tabular
\let\endsi@Otabular=\endtabular
\newenvironment{sitabular}[1]{%
  \def\hline{\si@hline}%
  \ifsi@hhline\def\hhline{\si@hhline}\fi%
  \newcolumntype{|}{!{\si@vline}}%
  \begin{si@Otabular}{#1}%
}{%
  \end{si@Otabular}%
}
%    \end{macrocode}
%
% \subsection{\texttt{silongtable} environment if loaded}
%
% \DescribeEnv{silongtable}
% \smallskip
%
%    \begin{macrocode}
\@ifclassloaded{longtable}{}{%
  \global\let\si@OLT@hline=\LT@hline%
  \newcommand{\si@LT@hline}{%
    \noalign{\si@bol}%
    \si@OLT@hline%
    \noalign{\si@eol}%
  }%
  \let\si@Olongtable=\longtable%
  \let\endsi@Olongtable=\endlongtable%
  \newenvironment{silongtable}[1]{%
    \def\LT@hline{\si@LT@hline}%
    \def\si@Ohline{\hline}%
    \def\hline{\si@hline}%
    \ifsi@hhline\def\hhline{\si@hhline}\fi%
    \newcolumntype{|}{!{\si@vline}}%
    \begin{si@Olongtable}{#1}%
  }{%
    \end{si@Olongtable}%
  }%
}
%    \end{macrocode}
%\Finale
